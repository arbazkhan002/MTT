\documentclass{article}
\usepackage{amssymb,amsmath}
\usepackage{ifxetex,ifluatex}
\ifxetex
  \usepackage{fontspec,xltxtra,xunicode}
  \defaultfontfeatures{Mapping=tex-text,Scale=MatchLowercase}
\else
  \ifluatex
    \usepackage{fontspec}
    \defaultfontfeatures{Mapping=tex-text,Scale=MatchLowercase}
  \else
    \usepackage[utf8]{inputenc}
  \fi
\fi
\ifxetex
  \usepackage[setpagesize=false, % page size defined by xetex
              unicode=false, % unicode breaks when used with xetex
              xetex,
              colorlinks=true,
              linkcolor=blue]{hyperref}
\else
  \usepackage[unicode=true,
              colorlinks=true,
              linkcolor=blue]{hyperref}
\fi
\hypersetup{breaklinks=true, pdfborder={0 0 0}}
\setlength{\parindent}{0pt}
\setlength{\emergencystretch}{3em}  % prevent overfull lines
\textheight 8.5in
\addtolength{\topmargin}{-1in}

\title{Place Descriptions}
\author{Arbaz Khan}
\date{Progress Report \\ \today}
\begin{document}
\maketitle 

\section{Progress}

\subsection*{Annotation of the triplets}
  Each of the three phrases of the triplet were annotated with an
  alignment scale of 1 to 3, which basically checked the identifiability
  of that phrase in the description. Tags:

  \begin{enumerate}
  \item
    1 denotes that the phrase was exactly identified as it is in the
    sentence of the description where the corresponding triplet was
    found.
  \item
    2 denotes that the different words of the phrase could be
    identified in the sentence of the triplet but not the complete
    phrase as it is.
  \item
    3 denotes that one or more of the words of the phrase were not
    identified in the sentence of the triplet.
  \end{enumerate}


This helped getting familiar with the complexity of the interpretability
of the triplets in a description. The more 3s you have, the more
difficult would it be to interpret a triplet out of the description.

\subsection*{Sentence labels}
  An extra field was added in the triplets data containing the
  sentence\_id which was the counter index of the sentence in which the
  triplet was found.

  This helped in understanding the depth of the connections between the
  sentences. If a phrase of the triplet occurs as RO in a number of
  sentences, then we know how far the context gets carried.
\section{Future Work}
The difficulty in identifiability is expected to be taken care of by the
parser. So, it would be worthwhile to go through the sample outputs of
the parser provided by Felix and check the extent upto which it takes
care of the identifiability, esp. of the prepositions (in). Also to be
looked at is whether the context is carried foe.

\subsection*{Carrying the context with DFS}

It was observed that the contexts used in human descriptions goes in a
depth-first fashion to describe path intersections. It is natural to
describe one path as far as you could and then come back to the others,
rather than providing path descriptions in a parallel fashion. For
example, in the desctiption , ``{[}..{]} \textit{there are three main
alternative paths that you can use to head north. The central one is up
some stairs} {[}..{]}. \textit{The path will take you to the} {[}..{]} \textit{where there
is} {[}..{]} \textit{and a little to the east, the Union House. Union House is a
large building containing} {[}..{]}''. And then the context ends and the
describer switches back to the other branch to explore a new depth,
``\textit{The second path} {[}..{]} \textit{from the south entrance takes you a little
bit to the west and then switches north} {[}..{]}. \textit{Near this} {[}..{]},
\textit{you find a crepes stand. From this path }{[}..{]}''. (Here, words like
``this path'' , ``the road'' represent the current context). And then
again it switches back, ``\textit{The third path} {[}..{]}''.

Hence, it gives rise to an idea of parsing in a depth-first fashion so
that the context is taken care of by the tree. It also provides a way to
resolve the demonstrative pronouns like ``this building'' where it would
mean the current context, i.e. the root of the subtree in progress.

Sample runs of such a tree were studied and it was found to resolve some
indirect descriptions but on the whole, its success depends upon the
parsing of the human descriptions

\end{document}
