% THIS IS SIGPROC-SP.TEX - VERSION 3.1
% WORKS WITH V3.2SP OF ACM_PROC_ARTICLE-SP.CLS
% APRIL 2009
%
% It is an example file showing how to use the 'acm_proc_article-sp.cls' V3.2SP
% LaTeX2e document class file for Conference Proceedings submissions.
% ----------------------------------------------------------------------------------------------------------------
% This .tex file (and associated .cls V3.2SP) *DOES NOT* produce:
%       1) The Permission Statement
%       2) The Conference (location) Info information
%       3) The Copyright Line with ACM data
%       4) Page numbering
% ---------------------------------------------------------------------------------------------------------------
% It is an example which *does* use the .bib file (from which the .bbl file
% is produced).
% REMEMBER HOWEVER: After having produced the .bbl file,
% and prior to final submission,
% you need to 'insert'  your .bbl file into your source .tex file so as to provide
% ONE 'self-contained' source file.
%
% Questions regarding SIGS should be sent to
% Adrienne Griscti ---> griscti@acm.org
%
% Questions/suggestions regarding the guidelines, .tex and .cls files, etc. to
% Gerald Murray ---> murray@hq.acm.org
%
% For tracking purposes - this is V3.1SP - APRIL 2009

\documentclass{iitkthesis}
%\usepackage{algorithm2e}
%\usepackage{algpseudocode}
%\usepackage[pass,paperwidth=8.5in,paperheight=11in]{geometry}
%\usepackage[paperwidth=8.5in, paperheight=11in]{geometry}
%\newcommand{\commentSW}[1]{\textbf{SW --} \emph{#1} \textbf{-- SW}}

\begin{document}
%\pdfpagewidth=8.5in
%\pdfpageheight=11in

\title{Dialog-Based Location Unaware Wayfinding}
%\titlenote{(Does NOT produce the permission block, copyright information nor page numbering). For use with ACM\_PROC\_ARTICLE-SP.CLS. Supported by ACM.}}
%\subtitle{[Extended Abstract]
%\titlenote{A full version of this paper is available as
%\textit{Author's Guide to Preparing ACM SIG Proceedings Using
%\LaTeX$2_\epsilon$\ and BibTeX} at
%\texttt{www.acm.org/eaddress.htm}}}
%
% You need the command \numberofauthors to handle the 'placement
% and alignment' of the authors beneath the title.
%
% For aesthetic reasons, we recommend 'three authors at a time'
% i.e. three 'name/affiliation blocks' be placed beneath the title.
%
% NOTE: You are NOT restricted in how many 'rows' of
% "name/affiliations" may appear. We just ask that you restrict
% the number of 'columns' to three.
%
% Because of the available 'opening page real-estate'
% we ask you to refrain from putting more than six authors
% (two rows with three columns) beneath the article title.
% More than six makes the first-page appear very cluttered indeed.
%
% Use the \alignauthor commands to handle the names
% and affiliations for an 'aesthetic maximum' of six authors.
% Add names, affiliations, addresses for
% the seventh etc. author(s) as the argument for the
% \additionalauthors command.
% These 'additional authors' will be output/set for you
% without further effort on your part as the last section in
% the body of your article BEFORE References or any Appendices.
% \setcoguide{Prof. Bharat Lohani}
% setexguide{Prof. Stephan Winter}
% setcoguidedept{Department of Geoinformatics}
% setexguidedept{Department of Geomatics}

\author{Arbaz Khan}
\iitbdegree{Btech-Mtech Integrated}
\department{Computer Science and Engineering}
\rollnum{Y9227128}
\setguide{Prof Harish Karnick}
\setguidedept{Computer Science and Engineering}
\setcoguide{Prof Bharat Lohani}
\setexguide{Prof. Stephan Winter}
\setcoguidedept{Department of Geoinformatics}
\setexguidedept{Department of Geomatics}
\setexguideaff{University of Melbourne}
\dissertation
\maketitle
\makecertificate
\begin{abstract}
\iffalse
  A computational model of understanding place descriptions is a cardinal 
issue in multiple disciplines and provides critical applications 
especially in dialog-driven geolocation services. This research targets 
the automated extraction of spatial triplets to represent qualitative 
spatial relations between recognized places from natural language place 
descriptions via a simple class of locative expressions. We attempt to 
produce triplets, informative and \textit{convenient} enough as a medium 
to convert verbal descriptions to graph representations of places and 
their relationships. We present a reasoning approach devoid of any 
external resources (such as maps, path geometries or robotic vision) for 
understanding place descriptions. We then apply our methodologies to 
situated place descriptions and study the results, its errors and implied 
future research.

{\bf *** This abstract is far too obscure. What are spatial triplets?
Which locative expressions? Where is a verbal description converted to
a graph representation? You are using a map and a database of landmarks.
Frankly, I dont understand this abstract. Write the abstract in clear
and simple language saying exactly what you have done. Dont borrow
unecessary terminology. ***}
\fi
%We also provide an insight into the complexity of the untackled problem of resolving the frame of reference.
\end{abstract}
\tableofcontents
\chapter{Introduction}

{\bf *** Start with a statement of the problem in simple language. Notions
and terminology can come after that. ***}

\section{Notions and Terminology}
\subsection{Wayfinding}
Human wayfinding is the process of purposeful and directed movement from 
an origin to a specific destination. It is different from spatial 
exploration or \textit{locomotion}, the other form of navigation where 
the goal is not to reach a specific destination but to contribute to 
cognitive map formation of an environment. So the daily trips that a 
person makes to his work-place from his home is a wayfinding task, while 
exploring an unfamiliar neighborhood in the town is locomotion or 
exploration. The problem of wayfinding is: identify the ordered 
sequence of actions that must be performed in a spatial environment to reach 
a desired location. For a car driver in a street network, these set of 
actions are related to determining the turning behaviour at every 
intersection that he encounters. For a person in a museum looking for a 
specific art form, these set of actions would be a sequence of hallways 
he needs to walk through, to get to the intended location. 

Further, it's easy to see that in our everyday interaction with space 
we are often involved in wayfinding tasks. In some, the exact sequence of 
actions is very familiar but in 
others we need either external assistance or a personal strategy to 
find the way. The former requires acquisition of spatial knowledge 
of the environment either through prior experience or through static 
information learned from maps and/or other media-based resources and so 
is prone to errors. Similarly, personal strategies used in wayfinding do 
not guarantee success in reaching the destination. Traditionally, 
people have used maps and compasses as external tools for 
wayfinding. These guidance instruments have evolved over the years to 
mobile navigation systems as the need was to provide incremental 
instructions as the person moves in space. This has 
proved more effective as the information on what needs to be known 
is provided only when needed. This approach shapes the 
modern form of assistance - \textit{location-aware} wayfinding. 
\subsection{Location Awareness}
A service is said to be \textit{location aware} if it allows an user to 
discover and communicate his position in the real world using some  
form of external hardware support. Location awareness has become 
a key component in many mobile computing applications \cite{parctab}. 
In the context of navigation, we can say that if the end-device knows its 
geographical location the service is location-aware. This definition 
comes from the realm of location-aware computing where location-awareness 
means ability to provide services based on the geographical location of a 
mobile device. Primarily, there are three different techniques for 
location sensing - Triangulation, Scene Analysis and Proximity Sensing 
\cite{hightower}. In modern location-aware services GPS is the 
most widely used because of its better relative accuracy. The Invisible 
Ideas Project \cite{perry} was the first of its kind to use flash and GPS 
technology to provide location-aware services.

It is easy to see how location-awareness can help in way-finding. 
With the help of techniques for location-awareness a mobile device is
able to determine its geographical location using sensing technology (
such as GPS) and then relay this information (along with any 
identity-based data like 
user ID, device ID, etc.) to the service provider. Thus, it can be a platform 
for realizing an effective wayfinding assistance system which can provide 
incremental delivery of instructions to the user based on the location of 
the user. 

\section{Background}
Modern advancements in technology have diminished the diffiulty 
in discovering and implementing wayfinding strategies. With the 
advent of smartphones, it is possible to build powerful 
applications which can show maps and compute routes based on preferences. 
It has removed the inconvenience of carrying a map and understanding its 
symbols and notation. The navigation services are dynamic, that is the 
representation of graphical information keeps changing with respect to 
the current information about the user. This is where location-awareness 
comes into the picture and applications have utilized the power of this 
facility to customize their functionality to maximally benefit the users. 
The quality of user experience has evolved over a series of developments 
with offerings like 3-D representation of the environment showing up 
places nearby for better orientation. Significant effort has been put 
into the research and development of systems for navigational 
assistance. With the techniques of augmented reality, it has been now 
made possible to attach digital information (such as images, voice notes) 
to the environment. Furthermore, local information on weather and traffic 
has been incorporated into these applications to further enhance an 
interactive wayfinding environment.

%{For ineffectiveness of GPS, refer to PLace Lab}
Despite the powerful services offered by digital navigational aids, there 
are some problems as well. When a service is location aware, 
the end-device needs to have typically high processing speeds to process 
communicated information. Apart from 
the installation and usage costs associated with these services, the 
limited accuracy of these positioning systems is a vital concern. The 
inaccuracies extend from measurement errors in positioning systems from 
satellites to those in the map-matching algorithms \cite{white2000} that 
attempt to associate recorded GPS data points\footnote{These associations 
are unreliable to an extent and usually mismatch in dense street networks 
with diverging roadways, overpasses and underpasses} with the correct 
roadway. The GPS sensors suffer from poor service coverage and need clear 
vision to the sky to allow location locking. Thus, GPS limitations extend 
to subways, underpasses, indoor environments and dense street networks 
with tall buildings. Furthermore, deploying a location sensor like GPS 
has overheads of cost and power consumption.

The other limitation associated with modern day wayfinding services is 
their utter dependence on the extrinsic information for providing route 
assistance. The quality of route instructions are based upon the details 
in the map and the available landmark information through crowdsourced 
databases. This availability is fairly non-uniform and the success in 
providing utility is highly variable. Its easy to find regions where the 
route instructions in these commercial applications have no incorporation 
of existing landmarks (due to their unrecorded entry in the spatial 
databases), and are merely turn-based. Also, despite the availability of 
the landmark information doesn't guarantee good quality route 
instructions unless the representational names used are consistent and 
recognised universally or locally. For example, since not all streets in 
India have names, various popular mapping services and applications which 
rely upon street names to convey route instructions, had to invent their 
own street naming conventions and thus are observably ineffective in the 
context of navigation assistance.

Beyond these limitations, most of the services require an internet 
connection between the server and the client particularly for 
delivering map data or route based information. Streaming such  
information requires good internet connectivity. Storing offline maps is 
an alternative option but may not be always feasible.   

\section{Motivation}
A pure voice-based way-finding cum location-tracking 
model would eliminate the high-end device requirement. Here we discuss two
major reasons for such a way-finding model. 
\subsection{Why design a dialog-based system?}
In addition to the above shortcomings in modern digital navigation 
services, research \cite{jensen2010,dalton,reagan2006} has 
identified user preference for auditory assistance as well as a memory 
advantage for auditory over visual information. It has been observed 
\cite{baldwin2009,furukawa2004} that auditory guidance facilitates the 
task of way-finding whereas visual guidance facilitates cognitive map 
formation. Ego-centeric auditory instructions (i.e., based on the driver'
s perception) reduce the workload in navigation for drivers who are 
involved more in the task of route learning rather than cognitive map 
formation. Besides this, a graphical interface is likely to seriously 
interfere with driving while talking or listening on a hands-free mobile 
device is less intrusive. Wayfinding 
can be treated as one of those tasks that demand high level of attention 
to avoid any road-side risks. Auditory route instructions have been 
observed to be processed and followed without interference to the driving 
task \cite{jensen2010,dalton}.

Experiments \cite{srinivasaneffect} have also indicated an additional 
benefit associated with auditory guidance. The reaction times are 
faster with pure auditory route instructions compared to electronic 
route maps or turn-by-turn displays. 

Furthermore,  one major advantage of a dialog system 
is that context-identification can be implemented by studying the 
response of the user to the instructions/questions. This helps in 
personalising the service to navigational 
strategies preferred by a user. Most navigation assistance 
technologies are still visually dominated and the use of audio has 
little researched. In the literature we have not foundo comparable research 
that has exploited the usefulness of plain voice dialog for 
navigation purposes.
\subsection{Why choose to be location-unaware?}
Most end user mobile devices (typically cell phones)  in India are low cost 
vanilla voice phones that also be used for texting. They do not have a GPS or
other location sensors.

%Also, it has been suggested in research \cite{itsbarry} that in a joint activity where there is a coordination to achieve a private/personal goal, spoken dialog interactions are more effective than typed conversations and has been put to use in designing intelligent tutoring systems. This indication serves a clue that reading out the route instructions should be more effective as compared to text-based communication. 

 \section{Objectives of the work}
We target the design of a \textit{Dialog-based, location unaware way-finding
} model which exploits the auditory mode of route guidance and eliminates 
the need for global positioning systems. A service is location-unaware if 
the end-device has no location sensitivity and can't determine its 
physical location either by itself (e.g.,GPS) or by sensing the resources 
from the environment (e.g., Infrared or wireless media). To further 
define a \textit{location-unaware} service, we describe it as a service 
under a pseudo location-aware model where one can determine the 
location of the end-user only approximately within a certain tolerance
without employing any location identification hardware support from the 
end-user device (in our case a cell phone). 

In this thesis we have conceptualized, designed and implemented
{
\textit{a dialog-based way-finding system with pseudo location-awareness.}
}

More concretely the objectives are:

First, to communicate route instructions, we have designed a cross-
lingual way-finding platform which gives instructional output in a simple
formal language. This formal language can be easily translated to
any natural language using a natural language generation module.

Second, we use a dialog based approach that uses intrinsic and extrinsic
landmarks to localize a user during the entire way-finding session.

Third, we aim to minimize the dependence of the system on any extrinsic 
information. By extrinsic information, we mean any information that is 
not readily and uniformly available due to the needs of large-scale 
manual processing e.g., crowdsourcing, or database built from a place 
directory/gazeteer (gazeteers are not available for all places). The 
intention is to maximize the use of inherent characteristics of an 
environment like the height and color of buildings, the existence of an 
open space and hence its shape, surrounding vegetation, etc. Such 
characteristics can be automically extracted from standard detection 
systems like LIDAR.

Fourth, we plan to provide a pure speech-based service which demands 
nothing more than an auditory medium for communicating with the mobile end-
user. Thus the the service functions independent of 
support for any wireless technology (such as bluetooth) or web access.
  \section{Roadmap}
Our design for way-finding uses dialog-based conversations to localize a 
user and guide him to the destination. To make the model generic, we design 
a communication protocol based on a formal representation of route instructions
to convey the path to the destination. The algorithm generates
a structured interaction for tracking the user's 
location in terms of en-route landmarks encountered. Chapter 3 introduces 
the semantics of the communication protocol as well as the the guidance 
algorithm to generate dynamic prompts for location tracking. 
In \textit{Chapter 4}, we present the localization algorithm to determine user's location by generating prompts for location tracking. It also discusses on the strategy adopted to reorient a user disoriented from his path due to possible misinterpretation of route instructions or because of an erroneous behaviour. 
Chapter 5 discusses the key implementation-specific aspects of the 
model which was used to study the effectiveness of the algorithm. To 
evaluate `goodness' of the algorithm, a simulation platform was set up to 
model a user with homogenous and non-homogenous speed patterns and 
erroneous behaviour in following the route instructions. Chapter 6 elaborates 
upon the employed user-modelling and introduces the goodness metrics used for 
evaluating the results of performance of the route guidance model. We end in 
Chapter 7 by providing a brief summary of the contributions made by the work
and an outlook on possible improvements of the model.




nd thus create no negative 
impact on reaction times \cite{srinivasaneffect}. 
 
At present, there is a multitude of systems specializing in spoken 
dialogue-systems for wayfinding, but a major portion of these target indoor 
wayfinding tasks. 
 \\ $<$More on indoor dialog systems, human-computer interaction and its difference with human-human and context specific wayfinding$>$
\fi
 \section{Driver modelling}
 \chapter{Route Guidance}
 \section{Communication Protocol}
 \section{Route Guidance Algorithm}
 \chapter{Implementation}
 \section{Data}
 \section{Preprocessing}
 \section{Knowledge Base}
 \chapter{Evaluation}
 \section{Simulation Design}
 \section{Goodness Metrics}
 \section{Results}
 \chapter{Conclusion and Future Work}
\bibliographystyle{plain}
\bibliography{references}
%\addcontents{toc}{bibliography}
\end{document}
