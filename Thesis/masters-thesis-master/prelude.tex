% prelude.tex
%   - titlepage
%   - dedication (optional)
%   - approval sheet
%   - course certificate
%   - table of contents, list of tables and list of figures
%   - nomenclature
%   - abstract
%============================================================================


\clearpage\pagenumbering{roman}  % This makes the page numbers Roman (i, ii, etc)



% TITLE PAGE
%   - define \title{} \author{} \date{}
\title{Functional SMT solving: A new interface for programmers}
\author{Siddharth Agarwal}
\date{June, 2012}

%  - Roll number, required for title page, approval sheet, and
%    certificate of course work 
\rollnum{Y7027429} 

%   - The default degree is ``Doctor of Philosophy''
%     (unless the document style msthesis is specified
%      and then the default degree is ``Master of Science'')
%     Degree can be changed using the command \iitbdegree{}
\iitbdegree{Master of Technology}

%   - The default report type is preliminary report.
%      * for a PhD thesis, specify \thesis
\thesis
%      * for a M.Tech./M.Phil./M.Des./M.S. dissertation, specify \dissertation
%\dissertation
%      * for a DIIT/B.Tech./M.Sc.project report, specify \project
%\project
%      * for any other type, use  \reporttype{}
%\reporttype{ReportType}

%   - The default department is ``Unknown Department''
%     The department can be changed using the command \department{}
\department{Computer Science \& Engineering}

%    - Set the guide's name
\setguide{Prof Amey Karkare}
\setguidedept{Department of Computer Science \& Engineering}

%   - once the above are defined, use \maketitle to generate the titlepage
\maketitle

%--------------------------------------------------------------------%
% CERTIFICATE
%     The first page after the title page.
\makecertificate

%--------------------------------------------------------------------%
% COPYRIGHT PAGE
%   - To include a copyright page use \copyrightpage
% \copyrightpage

%--------------------------------------------------------------------%
% ABSTRACT
\begin{abstract}
Satisfiability Modulo Theories (SMT) solvers are powerful tools that can quickly solve
complex constraints involving booleans, integers, first-order logic
predicates, lists, and other data types. They have a vast number of potential
applications, from constraint solving to program analysis and verification.
However, they are so complex to use that their power is inaccessible to all
but experts in the field.

We present an attempt to make using SMT solvers simpler by integrating the Z3 solver
into a host language, Racket. Our system defines a programmer's interface in
Racket that makes it easy to harness the power of Z3 to discover solutions to
logical constraints. The interface, although in Racket, retains the
structure and brevity of the SMT-LIB format. We demonstrate this using a range
of examples, from simple constraint solving to verifying recursive functions, all in
a few lines of code.
\end{abstract}

%--------------------------------------------------------------------%
% DEDICATION
%   Dedications, if any.
\begin{dedication}
To my grandfather
\end{dedication}

% Acknowledgements
\begin{acknowledgments}

This project would have never have even been started had it not been for my
thesis advisor Dr Amey Karkare's help and guidance. Right from the time we
were looking for ideas to when we finally had a concrete proposal in hand, his
support has been invaluable. He entertained all my crazy ideas, found which
ones had merit, and helped me build on them. I wouldn't have been able to do
this without him.

I'd like to thank my parents and the others in my life for being patient with
me, especially towards the final stretch of thesis completion when I wasn't at
my best socially.

The Racket community, at \texttt{\#racket} on Freenode IRC, helped me
understand some of the Racket syntax model's intricacies. That helped me avoid
dead ends both early on and towards the end.

Special thanks to Leonardo de Moura at Microsoft Research for taking the time
out to answer a few questions about Z3.

\end{acknowledgments}

%--------------------------------------------------------------------%
% CONTENTS, TABLES, FIGURES
\tableofcontents
\listoftables

\cleardoublepage
\phantomsection \label{listoffig}
\addcontentsline{toc}{chapter}{List of Figures}
\listof{program}{List of Figures}
%\listoffigures

\cleardoublepage\pagenumbering{arabic} % Make the page numbers Arabic (1, 2, etc)
